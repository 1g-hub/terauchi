%\documentstyle[epsf,twocolumn]{jarticle}       %LaTeX2e仕様
%\documentclass[twocolumn]{jarticle}     %pLaTeX2e仕様(platex.exeの場合)
\documentclass[onecolumn]{ujarticle}   %pLaTeX2e仕様(uplatex.exeの場合)
%%%%%%%%%%%%%%%%%%%%%%%%%%%%%%%%%%%%%%%%%%%%%%%%%%%%%%%%%%%%%%
%%
%%  基本バージョン
%%
%%%%%%%%%%%%%%%%%%%%%%%%%%%%%%%%%%%%%%%%%%%%%%%%%%%%%%%%%%%%%%%%
\setlength{\topmargin}{-45pt}
%\setlength{\oddsidemargin}{0cm}
\setlength{\oddsidemargin}{-7.5mm}
%\setlength{\evensidemargin}{0cm}
\setlength{\textheight}{24.1cm}
%setlength{\textheight}{25cm}
\setlength{\textwidth}{17.4cm}
%\setlength{\textwidth}{172mm}
\setlength{\columnsep}{11mm}

%\kanjiskip=.07zw plus.5pt minus.5pt


% 【節が変わるごとに (1.1)(1.2) … (2.1)(2.2) と数式番号をつけるとき】
%\makeatletter
%\renewcommand{\theequation}{%
%\thesection.\arabic{equation}} %\@addtoreset{equation}{section}
%\makeatother

%\renewcommand{\arraystretch}{0.95} 行間の設定
%%%%%%%%%%%%%%%%%%%%%%%%%%%%%%%%%%%%%%%%%%%%%%%%%%%%%%%%
%\usepackage{graphicx}   %pLaTeX2e仕様(\documentstyle ->\documentclass)
\usepackage[dvipdfmx]{graphicx}
\usepackage{subcaption}
\usepackage{multirow}
\usepackage{amsmath}
\usepackage{url}
\usepackage{ulem}
%%%%%%%%%%%%%%%%%%%%%%%%%%%%%%%%%%%%%%%%%%%%%%%%%%%%%%%%
\begin{document}

	%bibtex用の設定
	%\bibliographystyle{ujarticle}
	\noindent

	\hspace{1em}
	2020 年 4 月 17 日
	ゼミ資料
	\hfill
	M2 寺内 光

	\vspace{2mm}

	\hrule

	\begin{center}
		{\Large \bf 進捗報告}
	\end{center}


	\hrule
	\vspace{3mm}

	% ‚ここから 文章 Start!
	\section{今週やったこと}
	\begin{itemize}{
		\item{DEAPのドキュメンテーションを読んで要約}
		\item{Cythonの勉強}
	}\end{itemize}
	\subsection{DEAP のドキュメンテーションを読んで要約}
	リモートが切れていることに気づいて作ってた資料を乗せることができず...また申請するのでリモート繋ぐときに更新しておきます.
	基本的に GA の交差,選択,突然変異についてまとめました.なにか実装に関して良い資料等があれば教えてほしいです.

	\subsection{Cython の勉強}
	Cython は C/C++ で Python の拡張モジュールを作成するために開発されたプログラミング言語.Python に静的な型付けをすることによって高速化できる.NumPy, SciPy, scikit-learn, Pandas などが裏側では Cython によって実装されている.\sout{競プロのサイトで使えるようになったので}我々が普段お世話になっている NumPy の気持ちを知るために入門.Python の構文がそのまま感覚的に使え,ただ Cython のコンパイラを用いただけでも関数のオーバヘッドがなくなり,高速になるのはすごく便利.静的型付けによって特にループ処理が高速になり,これだけでもかなり C, C++ と Python のいいとこ取りという感じ.拡張モジュールとして書かれたヘッダファイルがそのまま include して使えるので一度拡張モジュールをまとめておくと簡単に外部から呼び出せる.Python の高速化に興味がある人はぜひ.

	また,この知見を活かして競技プログラミングに挑んだ結果,2020/4/17 時点でのすべての Cython 提出者の中で最も早いプラグラムよりも 2 倍以上も高速に動作するプログラムを書くことに成功した.
	\cite{atcoder}

	\subsection{Macbook を買いました}
	Quality of Remote Work が上がりました.

	\section{来週のタスク}
	GA の実装 or 4コマ.相談しつつ.

	% 参考文献リスト
	\bibliographystyle{unsrt}
	\bibliography{2020_04_17}
\end{document}
