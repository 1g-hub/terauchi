%\documentstyle[epsf,twocolumn]{jarticle}       %LaTeX2e仕様
%\documentclass[twocolumn]{jarticle}     %pLaTeX2e仕様(platex.exeの場合)
\documentclass[onecolumn]{ujarticle}   %pLaTeX2e仕様(uplatex.exeの場合)
%%%%%%%%%%%%%%%%%%%%%%%%%%%%%%%%%%%%%%%%%%%%%%%%%%%%%%%%%%%%%%
%%
%%  基本バージョン
%%
%%%%%%%%%%%%%%%%%%%%%%%%%%%%%%%%%%%%%%%%%%%%%%%%%%%%%%%%%%%%%%%%
\setlength{\topmargin}{-45pt}
%\setlength{\oddsidemargin}{0cm}
\setlength{\oddsidemargin}{-7.5mm}
%\setlength{\evensidemargin}{0cm}
\setlength{\textheight}{24.1cm}
%setlength{\textheight}{25cm}
\setlength{\textwidth}{17.4cm}
%\setlength{\textwidth}{172mm}
\setlength{\columnsep}{11mm}

%\kanjiskip=.07zw plus.5pt minus.5pt


% 【節が変わるごとに (1.1)(1.2) … (2.1)(2.2) と数式番号をつけるとき】
%\makeatletter
%\renewcommand{\theequation}{%
%\thesection.\arabic{equation}} %\@addtoreset{equation}{section}
%\makeatother

%\renewcommand{\arraystretch}{0.95} 行間の設定
%%%%%%%%%%%%%%%%%%%%%%%%%%%%%%%%%%%%%%%%%%%%%%%%%%%%%%%%
%\usepackage{graphicx}   %pLaTeX2e仕様(\documentstyle ->\documentclass)
\usepackage[dvipdfmx]{graphicx}
\usepackage{subcaption}
\usepackage{multirow}
\usepackage{amsmath}
\usepackage{url}
\usepackage[bb=boondox]{mathalfa}
\newcommand{\argmax}{\mathop{\rm arg~max}\limits}
\newcommand{\argmin}{\mathop{\rm arg~min}\limits}
%%%%%%%%%%%%%%%%%%%%%%%%%%%%%%%%%%%%%%%%%%%%%%%%%%%%%%%%
\begin{document}

	%bibtex用の設定
	%\bibliographystyle{ujarticle}
	\noindent

	\hspace{1em}
	2020 年 5 月 29 日
	ゼミ資料
	\hfill
	M2 寺内 光

	\vspace{2mm}

	\hrule

	\begin{center}
		{\Large \bf 進捗報告}
	\end{center}


	\hrule
	\vspace{3mm}

	% ‚ここから 文章 Start!
	\section{今週やったこと}
	\begin{itemize}{
		\item{パワポ修正}
		\item{TDGA の論文読み}
		\item{入れ替えテスト実験}
	}\end{itemize}

	\subsection{テスト実験}
	先週のゼミで話に上がっていた 4 コマの順列並び替え問題を試した.入れ替えていないものをクラス 0 としてその他の入れ替え方法 23 通りをクラス 1 とした.
	クラス間の個数の偏りはクラス間に100倍の重みを設定することで緩和している.また,エポックの内最も評価 F-1 スコアが高かったものを採用した.

	表 \ref{tab:cond_exp1} に実験条件を示す.

	\begin{table}[h]
		\centering
		\caption{実験条件}
		\vspace{-4mm}
		\label{tab:cond_exp1}
		\begin{tabular}{|c||c|} \hline
			最適化関数&Adam\\ \hline
			損失関数&重み付き交差エントロピー\\ \hline
			エポック&100\\ \hline
			訓練:検証&9:1\\ \hline
			クラス0(Train)&627件\\ \hline
			クラス1(Train)&14421件\\ \hline
		\end{tabular}
	\end{table}

	表 \ref{tab:result1} に実験結果を示す.
	\begin{table}[h]
		\vspace{1mm}
		\centering
		\caption{入れ替え識別結果}
		\vspace{-3mm}
		\label{tab:result1}
		\begin{tabular}{|c|c|c|c|c|} \hline
	    クラス&Precision&Recall&F-1&サンプル数\\ \hline\hline
	   	0(入れ替えなし)&0.09&0.04&0.06&72\\ \hline
	   	1(入れ替えあり)&0.96&0.98&0.97&1656\\ \hline
			macro avg&0.52&0.51&0.51&1728\\ \hline
			weighted avg&0.92&0.94&0.93&1728\\ \hline
			base line(macro F-1 avg)&-&-&0.49&\\ \hline
			accuracy&&&&0.90\\ \hline
		\end{tabular}
	\end{table}

	また,コマを逆順にしたものをクラス 1 にした実験も一応試した.
	表 \ref{tab:result2} に実験結果を示す.
	\begin{table}[h]
		\vspace{1mm}
		\centering
		\caption{逆順入れ替え識別結果}
		\vspace{-3mm}
		\label{tab:result2}
		\begin{tabular}{|c|c|c|c|c|} \hline
	    クラス&Precision&Recall&F-1&サンプル数\\ \hline\hline
	   	0(入れ替えなし)&0.54&0.60&0.57&72\\ \hline
	   	1(逆順入れ替え)&0.55&0.49&0.51&72\\ \hline
			macro avg&0.54&0.54&0.54&144\\ \hline
			base line&0.50&0.50&0.50&\\ \hline
			accuracy&&&&0.5416\\ \hline
		\end{tabular}
	\end{table}

	表 \ref{tab:result3} に追加でコマ順を(3, 4, 1, 2)と固定した実験結果を示す.
	\begin{table}[h]
		\vspace{1mm}
		\centering
		\caption{固定入れ替え識別結果}
		\vspace{-3mm}
		\label{tab:result3}
		\begin{tabular}{|c|c|c|c|c|} \hline
			クラス&Precision&Recall&F-1&サンプル数\\ \hline\hline
			0(入れ替えなし)&0.59&0.64&0.61&72\\ \hline
			1(固定入れ替え)&0.61&0.56&0.58&72\\ \hline
			macro avg&0.60&0.60&0.60&144\\ \hline
			base line&0.50&0.50&0.50&\\ \hline
			accuracy&&&&0.5972\\ \hline
		\end{tabular}
	\end{table}

	やはりあまり精度が出ないなぁという感じ.一応パラメータチューニングはしていないので多少は上がる可能性はあるが,モデルアーキテクチャか実験設定を見直す必要がありそう.

	ひとまずこの実験はここまでにして JSAI 後は TDGA 周りの実装を相談しつつ.

	\section{来週のタスク}
	JSAI 前の調整,準備.

	% 参考文献リスト
	% \bibliographystyle{unsrt}
	% \bibliography{2020_05_29}
\end{document}
