%\documentstyle[epsf,twocolumn]{jarticle}       %LaTeX2e仕様
%\documentclass[twocolumn]{jarticle}     %pLaTeX2e仕様(platex.exeの場合)
\documentclass[onecolumn]{ujarticle}   %pLaTeX2e仕様(uplatex.exeの場合)
%%%%%%%%%%%%%%%%%%%%%%%%%%%%%%%%%%%%%%%%%%%%%%%%%%%%%%%%%%%%%%
%%
%%  基本バージョン
%%
%%%%%%%%%%%%%%%%%%%%%%%%%%%%%%%%%%%%%%%%%%%%%%%%%%%%%%%%%%%%%%%%
\setlength{\topmargin}{-45pt}
%\setlength{\oddsidemargin}{0cm}
\setlength{\oddsidemargin}{-7.5mm}
%\setlength{\evensidemargin}{0cm}
\setlength{\textheight}{24.1cm}
%setlength{\textheight}{25cm}
\setlength{\textwidth}{17.4cm}
%\setlength{\textwidth}{172mm}
\setlength{\columnsep}{11mm}

%\kanjiskip=.07zw plus.5pt minus.5pt

% 【節が変わるごとに (1.1)(1.2) … (2.1)(2.2) と数式番号をつけるとき】
%\makeatletter
%\renewcommand{\theequation}{%
%\thesection.\arabic{equation}} %\@addtoreset{equation}{section}
%\makeatother

%\renewcommand{\arraystretch}{0.95} 行間の設定
%%%%%%%%%%%%%%%%%%%%%%%%%%%%%%%%%%%%%%%%%%%%%%%%%%%%%%%%
%\usepackage{graphicx}   %pLaTeX2e仕様(\documentstyle ->\documentclass)
\usepackage[dvipdfmx]{graphicx}
\usepackage{subcaption}
\usepackage{multirow}
\usepackage{amsmath}
\usepackage{url}
\usepackage[bb=boondox]{mathalfa}
\usepackage{listings}
\newcommand{\argmax}{\mathop{\rm arg~max}\limits}
\newcommand{\argmin}{\mathop{\rm arg~min}\limits}

\lstset{%
  language={Python},
  basicstyle={\small},%
  identifierstyle={\small},%
  commentstyle={\small\itshape},%
  keywordstyle={\small\bfseries},%
  ndkeywordstyle={\small},%
  stringstyle={\small\ttfamily},
  frame={tb},
  breaklines=true,
  columns=[l]{fullflexible},%
  numbers=left,%
  xrightmargin=0zw,%
  xleftmargin=3zw,%
  numberstyle={\scriptsize},%
  stepnumber=1,
  numbersep=1zw,%
  lineskip=-0.5ex%
}

%%%%%%%%%%%%%%%%%%%%%%%%%%%%%%%%%%%%%%%%%%%%%%%%%%%%%%%%
\begin{document}

	%bibtex用の設定
	%\bibliographystyle{ujarticle}
	\noindent

	\hspace{1em}
	2020 年 10 月 23 日
	ゼミ資料
	\hfill
	M2 寺内 光

	\vspace{2mm}

	\hrule

	\begin{center}
		{\Large \bf 進捗報告}
	\end{center}

	\hrule
	\vspace{3mm}

	% ‚ここから 文章 Start!
	\section{今週やったこと}
	拡張の転移可能性 および epoch 数増加に対する accuracy の増分に関する調査

  \section{結果}
  CIFAR-10+WRN28-2 で探索した拡張を WRN28-10 に適用する実験および探索時の epoch は変えずに最終訓練の epoch を増やす実験をした.また,各識別率はいずれも 1 回試行におけるものである.

  表 \ref{tab:compare_experiments} に結果を示す.
  \begin{table}[ht]
		\centering
		\caption{Test accuracy(\%)}
		\label{tab:compare_experiments}
		\begin{tabular}{l||c c c c c c} \hline
		  &Base(200 epoch)&Base(300 epoch)&RA&200 epoch&300 epoch&400 epoch\\ \hline
			CIFAR-10 &&&&&&\\
      WRN28-2&94.9&94.98&95.8&95.89&96.08&96.14\\
			WRN28-10&96.1&96.23&97.3&97.10&97.26&97.30\\
      WRN28-10(transfer)&-&-&-&97.13&97.20&97.34\\ \hline
      SVHN-CORE &&&&&&\\
      WRN28-10&96.9&97.22&98.3&97.90&97.96&- \\
		\end{tabular}
	\end{table}

  1 回試行なのでぶれは大きいが, epoch の増加に対して精度が向上していることが確認できる.また,拡張を転移したほうが精度が(微妙に)あがっているが,誤差の範囲ともとれる.


  \section{〜feedback(11/3)のタスク}
  漫画データセットに対する適用実験

	% 参考文献リスト
	% \bibliographystyle{unsrt}
	% \bibliography{2020_10_23}
\end{document}
