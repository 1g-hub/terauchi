%\documentstyle[epsf,twocolumn]{jarticle}       %LaTeX2e仕様
%\documentclass[twocolumn]{jarticle}     %pLaTeX2e仕様(platex.exeの場合)
\documentclass[onecolumn]{ujarticle}     %pLaTeX2e仕様(uplatex.exeの場合)
%%%%%%%%%%%%%%%%%%%%%%%%%%%%%%%%%%%%%%%%%%%%%%%%%%%%%%%%%%%%%%
%%
%%  基本バージョン
%%
%%%%%%%%%%%%%%%%%%%%%%%%%%%%%%%%%%%%%%%%%%%%%%%%%%%%%%%%%%%%%%%%
\setlength{\topmargin}{-45pt}
%\setlength{\oddsidemargin}{0cm} 
\setlength{\oddsidemargin}{-7.5mm}
%\setlength{\evensidemargin}{0cm} 
\setlength{\textheight}{24.1cm}
%setlength{\textheight}{25cm} 
\setlength{\textwidth}{17.4cm}
%\setlength{\textwidth}{172mm} 
\setlength{\columnsep}{11mm}

%\kanjiskip=.07zw plus.5pt minus.5pt


% 【節が変わるごとに (1.1)(1.2) … (2.1)(2.2) と数式番号をつけるとき】
%\makeatletter
%\renewcommand{\theequation}{%
%\thesection.\arabic{equation}} %\@addtoreset{equation}{section}
%\makeatother

%\renewcommand{\arraystretch}{0.95} 行間の設定

%%%%%%%%%%%%%%%%%%%%%%%%%%%%%%%%%%%%%%%%%%%%%%%%%%%%%%%%
\usepackage{graphicx}   %pLaTeX2e仕様(\documentstyle ->\documentclass)
%%%%%%%%%%%%%%%%%%%%%%%%%%%%%%%%%%%%%%%%%%%%%%%%%%%%%%%%
\begin{document}
	
%bibtex用の設定
%\bibliographystyle{ujarticle} 

\noindent

\hspace{1em}
2019/2/21
ゼミ資料
\hfill
B3 寺内 光

\vspace{2mm}

\hrule

\begin{center}
{\Large \bf 進捗報告}
\end{center}


\hrule
\vspace{3mm}

% ‚ここから 文章 Start!
\section{現状(JSAI の段階でできていること)}
	\begin{itemize}
		\item 4 コマ画像を CAE で分散表現化,および分散表現を用いた識別
		\item パーツ(口,目)を抜いた画像の生成
		\item t-SNEによる分散表現のプロット
	\end{itemize}
\section{Data Augmentation}
	CAE の入力に keras の ImageDataGenerator を使って Data Augmentation をしてRandom Forestを用いた識別実験をしてみたが,識別率の向上は見られなかった.識別器側でデータを拡張する手法があればいいかもしれない.
	また,補完方法には nearest を用いた.
	
	\begin{table}[h]
		\centering
		\caption{Data Augmentation識別実験}
		\begin{tabular}{|c|c|} \hline
			拡張設定&識別率\\ \hline\hline
			そのまま使用&0.813\\ \hline
			左右反転&0.771\\ \hline
			回転( 限度10°)&0.667\\ \hline
			平行移動( 限度10\% )&0.625\\ \hline
			上記すべて&0.500\\ \hline
		\end{tabular}
	\end{table}

\section{今後の予定}
	\begin{itemize}
		\item 他のパーツ抜き画像に対する識別実験
		\item パーツ抜きの分散表現の変化の観察
		\item 中間層出力を縮小(画像のリサイズ)
	\end{itemize}


\end{document}
