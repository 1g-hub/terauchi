%\documentstyle[epsf,twocolumn]{jarticle}       %LaTeX2e仕様
%\documentclass[twocolumn]{jarticle}     %pLaTeX2e仕様(platex.exeの場合)
\documentclass[onecolumn]{ujarticle}   %pLaTeX2e仕様(uplatex.exeの場合)
%%%%%%%%%%%%%%%%%%%%%%%%%%%%%%%%%%%%%%%%%%%%%%%%%%%%%%%%%%%%%%
%%
%%  基本バージョン
%%
%%%%%%%%%%%%%%%%%%%%%%%%%%%%%%%%%%%%%%%%%%%%%%%%%%%%%%%%%%%%%%%%
\setlength{\topmargin}{-45pt}
%\setlength{\oddsidemargin}{0cm} 
\setlength{\oddsidemargin}{-7.5mm}
%\setlength{\evensidemargin}{0cm} 
\setlength{\textheight}{24.1cm}
%setlength{\textheight}{25cm} 
\setlength{\textwidth}{17.4cm}
%\setlength{\textwidth}{172mm} 
\setlength{\columnsep}{11mm}

%\kanjiskip=.07zw plus.5pt minus.5pt


% 【節が変わるごとに (1.1)(1.2) … (2.1)(2.2) と数式番号をつけるとき】
%\makeatletter
%\renewcommand{\theequation}{%
%\thesection.\arabic{equation}} %\@addtoreset{equation}{section}
%\makeatother

%\renewcommand{\arraystretch}{0.95} 行間の設定
%%%%%%%%%%%%%%%%%%%%%%%%%%%%%%%%%%%%%%%%%%%%%%%%%%%%%%%%
%\usepackage{graphicx}   %pLaTeX2e仕様(\documentstyle ->\documentclass)
\usepackage[dvipdfmx]{graphicx}
\usepackage{subcaption}
\usepackage{multirow}
\usepackage{amsmath}
\usepackage{url}
\usepackage{ulem}
%%%%%%%%%%%%%%%%%%%%%%%%%%%%%%%%%%%%%%%%%%%%%%%%%%%%%%%%
\begin{document}
	
	%bibtex用の設定
	%\bibliographystyle{ujarticle} 
	\noindent
	
	\hspace{1em}
	2019 年 10 月 18 日
	ゼミ資料
	\hfill
	M1 寺内 光
	
	\vspace{2mm}
	
	\hrule
	
	\begin{center}
		{\Large \bf 進捗報告}
	\end{center}
	
	
	\hrule
	\vspace{3mm}
	
	% ‚ここから 文章 Start!
	\section{今週やったこと}
	\begin{itemize}
		\item Deeplabv3+のモデルを用いてデータセット204枚で実験を回した.
	\end{itemize}

	\subsection{Deeplabv3+のモデルを用いてデータセット204枚で実験を回した.}
	先週に引き続き,今度はDeeplabv3+のモデルを用いてフルのデータセットで学習をした.訓練:テスト=95:5と設定した.初期重みから学習をはじめ,5000エポックの学習を終えてもすべてのPredictは全体が黒色(背景ラベル)の画像を出力するだけであった.これには以下のような複数の原因が考えられた.
	
	\begin{itemize}
		\item 目の領域が小さすぎる問題(クラス数の偏り)
		\item 単に学習不足
		\item データセット不足の問題
	\end{itemize}

	\subsubsection{目の領域が小さすぎる問題(クラス数の偏り)}
	これに対しては損失関数の実装にクラスごとに重みを設定するコードを追加することで対応した.目のラベルに対して大きな重みをつけると全体が目のラベルとして出力されることを確認した.しかし,正確に目の領域が検出されるわけではなく,まだ不十分である.

	\subsubsection{単に学習不足}
	モデルがかなり複雑でパラメータも多いので初期重みから学習を始めるとエポックが足りない可能性があった(train画像に対するPredictも黒色画像).そのために対照実験としてpascal vocのデータセットを用いて初期重みから同じように5000エポック回してみた.結果は全体が背景として出力されていたので.学習不足か学習コードの設定ミスの可能性が出てきた.
	
	\subsubsection{データセット不足の問題}
	これに関してはまだ不明で,まずはpascal vocを用いた実験がうまくいってから枚数がネックになっているようならオーギュメンテーションをかける方針にしたい.

	\section{今後の方針}
	まずはエポックを増やした実験がうまくいくか確認する.現在5000エポックで1時間ほどで回り終わるのでエポックを増やして回しながら様子見をする.ダメそうならまた別の路線を考えないといけないかもしれない.

	
\end{document}