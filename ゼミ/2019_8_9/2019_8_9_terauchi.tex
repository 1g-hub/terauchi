%\documentstyle[epsf,twocolumn]{jarticle}       %LaTeX2e仕様
%\documentclass[twocolumn]{jarticle}     %pLaTeX2e仕様(platex.exeの場合)
\documentclass[onecolumn]{ujarticle}   %pLaTeX2e仕様(uplatex.exeの場合)
%%%%%%%%%%%%%%%%%%%%%%%%%%%%%%%%%%%%%%%%%%%%%%%%%%%%%%%%%%%%%%
%%
%%  基本バージョン
%%
%%%%%%%%%%%%%%%%%%%%%%%%%%%%%%%%%%%%%%%%%%%%%%%%%%%%%%%%%%%%%%%%
\setlength{\topmargin}{-45pt}
%\setlength{\oddsidemargin}{0cm} 
\setlength{\oddsidemargin}{-7.5mm}
%\setlength{\evensidemargin}{0cm} 
\setlength{\textheight}{24.1cm}
%setlength{\textheight}{25cm} 
\setlength{\textwidth}{17.4cm}
%\setlength{\textwidth}{172mm} 
\setlength{\columnsep}{11mm}

%\kanjiskip=.07zw plus.5pt minus.5pt


% 【節が変わるごとに (1.1)(1.2) … (2.1)(2.2) と数式番号をつけるとき】
%\makeatletter
%\renewcommand{\theequation}{%
%\thesection.\arabic{equation}} %\@addtoreset{equation}{section}
%\makeatother

%\renewcommand{\arraystretch}{0.95} 行間の設定
%%%%%%%%%%%%%%%%%%%%%%%%%%%%%%%%%%%%%%%%%%%%%%%%%%%%%%%%
%\usepackage{graphicx}   %pLaTeX2e仕様(\documentstyle ->\documentclass)
\usepackage[dvipdfmx]{graphicx}
\usepackage{subcaption}
\usepackage{multirow}
\usepackage{amsmath}
\usepackage{url}
%%%%%%%%%%%%%%%%%%%%%%%%%%%%%%%%%%%%%%%%%%%%%%%%%%%%%%%%
\begin{document}
	
	%bibtex用の設定
	%\bibliographystyle{ujarticle} 
	\noindent
	
	\hspace{1em}
	2019 年 8 月 9 日
	ゼミ資料
	\hfill
	M1 寺内 光
	
	\vspace{2mm}
	
	\hrule
	
	\begin{center}
		{\Large \bf 進捗報告}
	\end{center}
	
	
	\hrule
	\vspace{3mm}
	
	% ‚ここから 文章 Start!
	\section{今週やったこと}
	google の official 版 DeepLab v3+ のサンプル \cite{deeplabv3_official} を動かせないか奮闘中.
	実装は \cite{cite1, cite2, cite3} を参考にしています.
	まずは PASCAL VOC 2012 を用いてサンプルが動作するか確かめたい.tensorflow の実装が理解できるかどうかはやってみないとわからないが...       
	
	\section{サンプルデータセット入手}
	PASCAL VOC 2012 \cite{pascal-voc-2012} を入手して整形する bash ファイルがエラーを吐く問題にずっと悩まされていました.Ubuntu の bash と dash の違いが原因っぽい...?とにかくデータセットは別で入手したので整形を今やっているところです.この辺りは上のサイトに丁寧に書いてあるので読めばわかりそう.
	                                                                                                                                             

	\section{夏休みのスケジュール}\noindent
	帰省 8/12-8/16\\
	日立インターン(東京) 8/21-9/12\\
	GREC パワポ等準備 9/13-9/18\\
	GREC(オーストラリア) 9/19(前日)-9/22\\
	の予定です.またインターンの不在に関しては周知します.
	
	% 参考文献リスト
	\bibliographystyle{unsrt}
	\bibliography{2019_8_9_terauchi}
\end{document}