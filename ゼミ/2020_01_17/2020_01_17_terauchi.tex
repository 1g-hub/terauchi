%\documentstyle[epsf,twocolumn]{jarticle}       %LaTeX2e仕様
%\documentclass[twocolumn]{jarticle}     %pLaTeX2e仕様(platex.exeの場合)
\documentclass[onecolumn]{ujarticle}   %pLaTeX2e仕様(uplatex.exeの場合)
%%%%%%%%%%%%%%%%%%%%%%%%%%%%%%%%%%%%%%%%%%%%%%%%%%%%%%%%%%%%%%
%%
%%  基本バージョン
%%
%%%%%%%%%%%%%%%%%%%%%%%%%%%%%%%%%%%%%%%%%%%%%%%%%%%%%%%%%%%%%%%%
\setlength{\topmargin}{-45pt}
%\setlength{\oddsidemargin}{0cm}
\setlength{\oddsidemargin}{-7.5mm}
%\setlength{\evensidemargin}{0cm}
\setlength{\textheight}{24.1cm}
%setlength{\textheight}{25cm}
\setlength{\textwidth}{17.4cm}
%\setlength{\textwidth}{172mm}
\setlength{\columnsep}{11mm}

%\kanjiskip=.07zw plus.5pt minus.5pt


% 【節が変わるごとに (1.1)(1.2) … (2.1)(2.2) と数式番号をつけるとき】
%\makeatletter
%\renewcommand{\theequation}{%
%\thesection.\arabic{equation}} %\@addtoreset{equation}{section}
%\makeatother

%\renewcommand{\arraystretch}{0.95} 行間の設定
%%%%%%%%%%%%%%%%%%%%%%%%%%%%%%%%%%%%%%%%%%%%%%%%%%%%%%%%
%\usepackage{graphicx}   %pLaTeX2e仕様(\documentstyle ->\documentclass)
\usepackage[dvipdfmx]{graphicx}
\usepackage{subcaption}
\usepackage{multirow}
\usepackage{amsmath}
\usepackage{url}
\usepackage{ulem}
%%%%%%%%%%%%%%%%%%%%%%%%%%%%%%%%%%%%%%%%%%%%%%%%%%%%%%%%
\begin{document}

	%bibtex用の設定
	%\bibliographystyle{ujarticle}
	% \twocolumn[
	\noindent

	\hspace{1em}
	2020 年 01 月 17 日
	ゼミ資料
	\hfill
	M1 寺内 光

	\vspace{2mm}

	\hrule

	\begin{center}
		{\Large \bf 進捗報告}
	\end{center}


	\hrule
	\vspace{3mm}
	% ]

	% ‚ここから 文章 Start!
	\section{今週やったこと}
	\begin{itemize}{
		\item{DCAI 原稿作成}
		\item{Manga109, 4コマストーリデータセットの整形}
		\item{単語出現頻度の確認}
	}
	\end{itemize}

	\subsection{DCAI 原稿作成}
	8 割くらい完成.残りは JSAI の abst 提出以降に完成させる予定.

	\subsection{Manga109, 4コマストーリデータセットの整形}
	Manga109 のうち,4 コマ漫画を抜き出して画像とセリフが対応している json ファイルを作成.BERT(事前学習済み)を用いて得たセリフの分散表現データも同様に作成.

	表 \ref{tab:num_dataset} に用いたタイトルと各タイトルの画像枚数を示す(セリフがないものあり,かっこ内はノイズデータ削除後の件数).

	\begin{table}[h]
		\vspace{-3mm}
		\centering
		\caption{タイトルと件数}
		\label{tab:num_dataset}
		\begin{tabular}{|c|c|} \hline
			タイトル&件数\\ \hline\hline
			総件数&2932(2802)\\ \hline
			幼稚園ぼうえい組&368\\ \hline
			徹さん&715(585)\\ \hline
			OLランチ&831\\ \hline
			高校の人達&938\\ \hline
			少年漫画タッチ&80\\ \hline
		\end{tabular}
	\end{table}

	\subsection{単語分布}
	表 \ref{tab:word_count} にデータセットの単語出現分布を示す
	\begin{table}[h]
		\vspace{-3mm}
		\centering
		\caption{単語出現分布}
		\label{tab:word_count}
		\begin{tabular}{|c|c|c|c|c|} \hline
			タイトル&総単語数&平均単語数&標準偏差&最も出現頻度の高いもの3件\\ \hline\hline
			幼稚園ぼうえい組&1360&3.546&11.212&...(218), の(195), は(113)\\ \hline
			徹さん&2275&3.678&24.721&!(1010), の(318), を(260)\\ \hline
			OLランチ&2730&3.506&17.309&・(556), の(379), は(225)\\ \hline
			高校の人達&3343&4.576&23.386&!(718), の(561), に(354)\\ \hline
			少年漫画タッチ&302&2.838&4.192&...(33), の(31), ?(29)\\ \hline
		\end{tabular}
	\end{table}

	\section{来週の予定}\noindent
	JSAI の abst に向けて土日でひとまず実験を回す.
	終わり次第、DCAI.
	%
	% \bibliographystyle{unsrt}
	% \bibliography{2020_01_10_terauchi}

\end{document}
