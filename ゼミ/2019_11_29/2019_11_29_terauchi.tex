%\documentstyle[epsf,twocolumn]{jarticle}       %LaTeX2e仕様
%\documentclass[twocolumn]{jarticle}     %pLaTeX2e仕様(platex.exeの場合)
\documentclass[onecolumn]{ujarticle}   %pLaTeX2e仕様(uplatex.exeの場合)
%%%%%%%%%%%%%%%%%%%%%%%%%%%%%%%%%%%%%%%%%%%%%%%%%%%%%%%%%%%%%%
%%
%%  基本バージョン
%%
%%%%%%%%%%%%%%%%%%%%%%%%%%%%%%%%%%%%%%%%%%%%%%%%%%%%%%%%%%%%%%%%
\setlength{\topmargin}{-45pt}
%\setlength{\oddsidemargin}{0cm}
\setlength{\oddsidemargin}{-7.5mm}
%\setlength{\evensidemargin}{0cm}
\setlength{\textheight}{24.1cm}
%setlength{\textheight}{25cm}
\setlength{\textwidth}{17.4cm}
%\setlength{\textwidth}{172mm}
\setlength{\columnsep}{11mm}

%\kanjiskip=.07zw plus.5pt minus.5pt


% 【節が変わるごとに (1.1)(1.2) … (2.1)(2.2) と数式番号をつけるとき】
%\makeatletter
%\renewcommand{\theequation}{%
%\thesection.\arabic{equation}} %\@addtoreset{equation}{section}
%\makeatother

%\renewcommand{\arraystretch}{0.95} 行間の設定
%%%%%%%%%%%%%%%%%%%%%%%%%%%%%%%%%%%%%%%%%%%%%%%%%%%%%%%%
%\usepackage{graphicx}   %pLaTeX2e仕様(\documentstyle ->\documentclass)
\usepackage[dvipdfmx]{graphicx}
\usepackage{subcaption}
\usepackage{multirow}
\usepackage{amsmath}
\usepackage{url}
\usepackage{ulem}
%%%%%%%%%%%%%%%%%%%%%%%%%%%%%%%%%%%%%%%%%%%%%%%%%%%%%%%%
\begin{document}

	%bibtex用の設定
	%\bibliographystyle{ujarticle}
	\noindent

	\hspace{1em}
	2019 年 11 月 29 日
	ゼミ資料
	\hfill
	M1 寺内 光

	\vspace{2mm}

	\hrule

	\begin{center}
		{\Large \bf 進捗報告}
	\end{center}


	\hrule
	\vspace{3mm}

	% ‚ここから 文章 Start!
	\section{今週やったこと}
	TensorFlow から PyTorch への移行が一応完了した(オリジナルデータセットでのテストが完了した).
	また,ドキュメンテーションも行ったので誰でも(Docker環境さえあれば)簡単に動かせるはず.

	\subsection{問題点}
	PyTorch実装による Predict の精度が低い.ひとまず回ったレベルの試行ではあるが,TensorFlow(Google)の mIoU が 0.678 だったのに対して PyTorch ベースは 0.58程度である(いずれも初期重みを与えている).mIoU は 0.5 を超えると一般的にいい値であるとは言われていて 0.58 もそこまで悪い値ではないものの,やはり精度の低下が気になる.今週中に Pascal voc データセットに対する Pretrained な重みで比較実験しようと思っていたがサーバが空いてなかったため来週に行う.
	% 表 \ref{tab:tensorflow_vs_pytorch} に Pascal vocデータセット,4コマ漫画ストーリーデータセットに対する精度の比較を示す.PyTorch で実装されたモデルが Google のものと異なる or Google の初期重みが強い or パラメータチューニングが不十分である等が原因としては考えられる.表 \ref{tab:deeplab_vs_PSPNet} に PyTorch ベースの DeepLabv3+ と PSPNet の比較を示す.DeepLabv3+では backbone に xception, PSPNet では ResNet を用いている.損失関数はいずれもシンプルな Cross Entropyである.
	%
	% \begin{table}[h]
	% 	\centering
	% 	\caption{TensorFlow vs PyTorch}
	% 	\label{tab:tensorflow_vs_pytorch}
	% 	\begin{tabular}{|c||c|c|} \hline
	% 		データセット&TensorFlow&PyTorch\\ \hline
	% 		Pascal voc&0.822&0.5\\ \hline
	% 		4コマ&0.678&0.580\\ \hline
	% 	\end{tabular}
	% \end{table}
	%
	% \begin{table}[h]
	% 	\centering
	% 	\caption{TensorFlow vs PyTorch}
	% 	\label{tab:deeplab_vs_PSPNet}
	% 	\begin{tabular}{|c||c|c|} \hline
	% 		データセット&DeepLabv3+&PSPNet\\ \hline
	% 		Pascal voc&0.822&0.5\\ \hline
	% 		4コマ&0.678&0.580\\ \hline
	% 	\end{tabular}
	% \end{table}

	\subsection{Auto-DeepLab}
	Semantic Segmentation のモデルを調べていると DeepLab の最新のモデルである Auto-DeepLab\cite{DBLP:journals/corr/abs-1901-02985}が発表されていることがわかった.
	Google が提供しているもの(ドキュメンテーション化はされてないので使い方が不明)や個人で作っているレポはいくつか見つけた(いずれも PyTorch 実装)ので時間を見つけて触りたい.
	初期重みからネットワーク探索およびセル探索を行うことで DeepLabv3+ やその他のモデルと同程度の性能を出しているので Pretrained なモデルによる精度の違いを解決できると期待できる.なお GPU メモリは 13GB 以上を要求されている模様...

	\section{来週のタスク}
	PyTorch ベースの DeepLabv3+ をパラメータチューニングしながら精度向上を目指す.

	% 参考文献リスト
	\bibliographystyle{unsrt}
	\bibliography{2019_11_29_terauchi}
\end{document}
