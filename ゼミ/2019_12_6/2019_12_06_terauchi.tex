%\documentstyle[epsf,twocolumn]{jarticle}       %LaTeX2e仕様
%\documentclass[twocolumn]{jarticle}     %pLaTeX2e仕様(platex.exeの場合)
\documentclass[onecolumn]{ujarticle}   %pLaTeX2e仕様(uplatex.exeの場合)
%%%%%%%%%%%%%%%%%%%%%%%%%%%%%%%%%%%%%%%%%%%%%%%%%%%%%%%%%%%%%%
%%
%%  基本バージョン
%%
%%%%%%%%%%%%%%%%%%%%%%%%%%%%%%%%%%%%%%%%%%%%%%%%%%%%%%%%%%%%%%%%
\setlength{\topmargin}{-45pt}
%\setlength{\oddsidemargin}{0cm}
\setlength{\oddsidemargin}{-7.5mm}
%\setlength{\evensidemargin}{0cm}
\setlength{\textheight}{24.1cm}
%setlength{\textheight}{25cm}
\setlength{\textwidth}{17.4cm}
%\setlength{\textwidth}{172mm}
\setlength{\columnsep}{11mm}

%\kanjiskip=.07zw plus.5pt minus.5pt


% 【節が変わるごとに (1.1)(1.2) … (2.1)(2.2) と数式番号をつけるとき】
%\makeatletter
%\renewcommand{\theequation}{%
%\thesection.\arabic{equation}} %\@addtoreset{equation}{section}
%\makeatother

%\renewcommand{\arraystretch}{0.95} 行間の設定
%%%%%%%%%%%%%%%%%%%%%%%%%%%%%%%%%%%%%%%%%%%%%%%%%%%%%%%%
%\usepackage{graphicx}   %pLaTeX2e仕様(\documentstyle ->\documentclass)
\usepackage[dvipdfmx]{graphicx}
\usepackage{subcaption}
\usepackage{multirow}
\usepackage{amsmath}
\usepackage{url}
\usepackage{ulem}
%%%%%%%%%%%%%%%%%%%%%%%%%%%%%%%%%%%%%%%%%%%%%%%%%%%%%%%%
\begin{document}

	%bibtex用の設定
	%\bibliographystyle{ujarticle}
	\noindent

	\hspace{1em}
	2019 年 12 月 6 日
	ゼミ資料
	\hfill
	M1 寺内 光

	\vspace{2mm}

	\hrule

	\begin{center}
		{\Large \bf 進捗報告}
	\end{center}


	\hrule
	\vspace{3mm}

	% ‚ここから 文章 Start!
	\section{今週やったこと}
	\begin{itemize}{
		\item{TensorFlow vs PyTorch 比較実験}
		\item{CAE リファクタリング}
		\item{青空文庫の作者推定(?)}
	}
	\end{itemize}

	\subsection{TensorFlow vs PyTorch 比較実験}
	PyTorchによるDeepLabv3+の精度が悪いという話があったのでTensorflowによるものと精度の比較を行った.
	用いたデータセットは pascal voc データセットで,エポックは十分に回して validation {\it mIoU} の高いものを Best Model として採用した.
	表 \ref{tab:tensorflow_vs_pytorch} に Pascal voc データセット,4コマ漫画ストーリーデータセットに対する精度の比較を示す.PyTorch で実装されたモデルが Google のものと微妙に異なる or Google の初期重みが強い(これが強そう) or パラメータチューニングが不十分である等が原因としては考えられる.また,Focal Loss を採用した実験も回したがあまり効果がなかった.

	\begin{table}[h]
		\centering
		\caption{TensorFlow vs PyTorch validation {\it mIoU}}
		\label{tab:tensorflow_vs_pytorch}
		\begin{tabular}{|c||c|c|c|} \hline
			データセット&TensorFlow(DeepLabv3+, xception)&PyTorch(Deeplabv3+, resnet)&PyTorch(PSPNet, resnet)\\ \hline
			Pascal voc&0.8335&0.7212&0.6959 \\ \hline
			4コマ&0.678&0.580&- \\ \hline
		\end{tabular}
	\end{table}

	\subsection{CAE リファクタリング}
	B3 の引き継ぎのためにCAEのドキュメンテーション化とリファクタリング中.
	来週の火曜日には渡せるようにします.

	\subsection{青空文庫の作者推定}
	就活のコーディング課題でなぜかごりごりのNLPタスクを解くはめに)今週のエネルギーはほとんどここに割かれた).
	来週の月曜の朝が締切なのでそこまではぼちぼちやります.
	BERT や Doc2Vec 周りの知見を得た.

	\section{来週のタスク}
	Auto-DeepLabを読めたのでちょくちょく触りつつ,
	PyTorch の DeepLabv3+ の実装をもう少し細部まで見る.

	\section{連絡事項}
	12/16-12/20 はインターンシップで東京にいるため研究室を欠席します.

\end{document}
