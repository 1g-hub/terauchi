%\documentstyle[epsf,twocolumn]{jarticle}       %LaTeX2e仕様
%\documentclass[twocolumn]{jarticle}     %pLaTeX2e仕様(platex.exeの場合)
\documentclass[onecolumn]{ujarticle}   %pLaTeX2e仕様(uplatex.exeの場合)
%%%%%%%%%%%%%%%%%%%%%%%%%%%%%%%%%%%%%%%%%%%%%%%%%%%%%%%%%%%%%%
%%
%%  基本バージョン
%%
%%%%%%%%%%%%%%%%%%%%%%%%%%%%%%%%%%%%%%%%%%%%%%%%%%%%%%%%%%%%%%%%
\setlength{\topmargin}{-45pt}
%\setlength{\oddsidemargin}{0cm}
\setlength{\oddsidemargin}{-7.5mm}
%\setlength{\evensidemargin}{0cm}
\setlength{\textheight}{24.1cm}
%setlength{\textheight}{25cm}
\setlength{\textwidth}{17.4cm}
%\setlength{\textwidth}{172mm}
\setlength{\columnsep}{11mm}

%\kanjiskip=.07zw plus.5pt minus.5pt


% 【節が変わるごとに (1.1)(1.2) … (2.1)(2.2) と数式番号をつけるとき】
%\makeatletter
%\renewcommand{\theequation}{%
%\thesection.\arabic{equation}} %\@addtoreset{equation}{section}
%\makeatother

%\renewcommand{\arraystretch}{0.95} 行間の設定
%%%%%%%%%%%%%%%%%%%%%%%%%%%%%%%%%%%%%%%%%%%%%%%%%%%%%%%%
%\usepackage{graphicx}   %pLaTeX2e仕様(\documentstyle ->\documentclass)
\usepackage[dvipdfmx]{graphicx}
\usepackage{subcaption}
\usepackage{multirow}
\usepackage{amsmath}
\usepackage{url}
\usepackage{ulem}
%%%%%%%%%%%%%%%%%%%%%%%%%%%%%%%%%%%%%%%%%%%%%%%%%%%%%%%%
\begin{document}

	%bibtex用の設定
	%\bibliographystyle{ujarticle}
	\noindent

	\hspace{1em}
	2019 年 11 月 22 日
	ゼミ資料
	\hfill
	M1 寺内 光

	\vspace{2mm}

	\hrule

	\begin{center}
		{\Large \bf 進捗報告}
	\end{center}


	\hrule
	\vspace{3mm}

	% ‚ここから 文章 Start!
	\section{今週やったこと}
	tensorflow から pytorch への移行に関して
	\begin{itemize}
		\item{いい感じのレポを探す}
		\item{実験環境のコンテナ化}
		\item{Pytorchの基本的な記法確認}
	\end{itemize}
	を行った.

	\subsection{いい感じのレポを探す}
	当初使うと思っていたレポよりいい感じのものがあったのでここ \cite{pytorch_segmentation} を参考に環境を作りたい.
	レポの特徴としては
	\begin{itemize}
		\item{複数の Semantic Segmentation のモデルが使える(11種類)(すごい)}
		\item{データセットもいろいろなものに対応している}
		\item{ロスのカスタマイズが可能(というかもう Focal Loss が実装されている)}
		\item{ネットワークの backbone が選択できる}
	\end{itemize}
	などなど,コードも整備されていて可読性も高く非常に感動を覚えた.

	\subsection{実験環境のコンテナ化}
	上のレポをフォークしてきて Docker のコンテナ化した環境でスクリプトが回るようにした(スペシャルサンクス:山本さん).
	ドキュメンテーション周りもちゃんとすれば相当使いやすいレポになりそう.

	\section{来週のタスク}
	サーバ GPU で pascal voc データセットを用いた実験が回るかどうか確認する.
	オリジナルデータセットへの適用方法を確認.

	来週のゼミは就活面接のため欠席します.

	% 参考文献リスト
	\bibliographystyle{unsrt}
	\bibliography{2019_11_22_terauchi}
\end{document}
