%\documentstyle[epsf,twocolumn]{jarticle}       %LaTeX2e仕様
%\documentclass[twocolumn]{jarticle}     %pLaTeX2e仕様(platex.exeの場合)
\documentclass[onecolumn]{ujarticle}   %pLaTeX2e仕様(uplatex.exeの場合)
%%%%%%%%%%%%%%%%%%%%%%%%%%%%%%%%%%%%%%%%%%%%%%%%%%%%%%%%%%%%%%
%%
%%  基本バージョン
%%
%%%%%%%%%%%%%%%%%%%%%%%%%%%%%%%%%%%%%%%%%%%%%%%%%%%%%%%%%%%%%%%%
\setlength{\topmargin}{-45pt}
%\setlength{\oddsidemargin}{0cm} 
\setlength{\oddsidemargin}{-7.5mm}
%\setlength{\evensidemargin}{0cm} 
\setlength{\textheight}{24.1cm}
%setlength{\textheight}{25cm} 
\setlength{\textwidth}{17.4cm}
%\setlength{\textwidth}{172mm} 
\setlength{\columnsep}{11mm}

%\kanjiskip=.07zw plus.5pt minus.5pt


% 【節が変わるごとに (1.1)(1.2) … (2.1)(2.2) と数式番号をつけるとき】
%\makeatletter
%\renewcommand{\theequation}{%
%\thesection.\arabic{equation}} %\@addtoreset{equation}{section}
%\makeatother

%\renewcommand{\arraystretch}{0.95} 行間の設定
%%%%%%%%%%%%%%%%%%%%%%%%%%%%%%%%%%%%%%%%%%%%%%%%%%%%%%%%
%\usepackage{graphicx}   %pLaTeX2e仕様(\documentstyle ->\documentclass)
\usepackage[dvipdfmx]{graphicx}
\usepackage{subcaption}
\usepackage{multirow}
\usepackage{amsmath}
\usepackage{url}
\usepackage{ulem}
%%%%%%%%%%%%%%%%%%%%%%%%%%%%%%%%%%%%%%%%%%%%%%%%%%%%%%%%
\begin{document}
	
	%bibtex用の設定
	%\bibliographystyle{ujarticle} 
	\noindent
	
	\hspace{1em}
	2019 年 10 月 4 日
	ゼミ資料
	\hfill
	M1 寺内 光
	
	\vspace{2mm}
	
	\hrule
	
	\begin{center}
		{\Large \bf 進捗報告}
	\end{center}
	
	
	\hrule
	\vspace{3mm}
	
	% ‚ここから 文章 Start!
	\section{今週やったこと}
	\begin{itemize}
		\item ラベル画像作成のためのスクリプト作成
		\item スクリプト及びシェルをオリジナルデータセットに適用できるように変更
	\end{itemize}
	
	\subsection{ラベル画像作成のためのスクリプト作成}
	Semantic Segmentationのためのラベル画像のフォーマットがかなり特殊で,RGB3チャンネルのすべての値を同じにすることでそれをインデックスとして扱うみたいなことをしている.この辺りの相性が透過 PNG と少し悪かったこともあり手こずった.ほとんど変換用のスクリプトはできたがまだ修正が必要(⇒今後の課題).
	
	\subsection{スクリプト及びシェルをオリジナルデータセットに適用できるように変更}
	実験用の Python スクリプトとシェルスクリプトは既存の PASCAL VOC データセットで学習するようになっていたので,オリジナルデータセットで学習できるように整備した.

	\section{今後の課題}
	\begin{itemize}
		\item ラベルをインデックスと紐づける
		\item PNG 画像の透過を消す
		\item フルのデータセットで回してみる
	\end{itemize}
	\subsection{ラベルをインデックスと紐づける}
	今エラーで止まっているのは(おそらく)これが原因で,現在 2 クラス(背景と目)で Semantic Segmentation を行おうとしてるが,その場合背景をRGB(0, 0, 0)として目をRGB(1, 1, 1)とする必要があるらしい.現在は見やすさのためにRGB(100, 100, 100)としていたので要修正.インデックスカラー周りの設定もしないといけないかもしれないので結果の可視化のためにこの辺りは見ておきたい.
	
	\subsection{PNG 画像の透過を消す}
	エラーは出てないがRGB3チャンネルにしてくれという旨の Warning が出ていたので要修正.一括で変換するスクリプトを作成する.
	
	\subsection{フルのデータセットで回してみる}
	現在は萌えタッチの 1 話のみ(6枚)を用いて試験的に回そうとしているのですべてのデータを乗せて一度回したい(まだ6枚のものも完全には回っていない).
	
\end{document}